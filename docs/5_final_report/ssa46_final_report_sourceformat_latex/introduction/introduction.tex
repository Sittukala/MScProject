\chapter{Introduction}

Personalized Buzzfeed is a news aggregator that gathers news all around the world and present it to the users considering and analysing users personal interests making it more personalized application. This application enables the users to explore all around the global news with a click anywhere on the globe thus giving the users enhanced visualization look and feel.

\section{Aims and Objectives}

The main Aim of this project is to gather worldwide news and provide them to users across the countries knowing their interests and likes. The three key objectives of the application are observed as Personalization, Visualization and Recommendation.


  
\begin{enumerate}  

 {\bf  \item  \large  Personalization} \newline
\hspace*{1cm} Personalized Buzzfeed app provides the users, news and information of their personal interests and likes and also stays up-to-date on their interests by analysing their day-to-day behaviour on the application. So It stays unique to individuals preferences anytime. \newline

{\bf Definition for Personalization} \newline
 The adaptivity to individual users or user groups in a website is called personalization. \newline
 Web Personalization is defined as the process of tailoring the content of a website to suit the user's needs and interests by making use of their behavioral activities in the application and their contextual details\cite{GARRIGOS2010991}. \newline
 
 
 
 {\bf  \item \large  Visualization}  \newline
\hspace*{1cm} Visualization is apparently achieved in this application providing a more appealing UI for users enhancing the user-friendliness. Visualization is provided in a more sophisticated way to the users in the form of more \textbf{interactive charts} and \textbf{geographical map} which is well achieved using amcharts library. \newline

{\bf  \item \large  Recommendation}  \newline
\hspace*{1cm} Recommendation is key feature which helps the users to handle increasing number of items or articles in case of news feed with the personalized suggestions provided by the application based on their preference patterns. According to research there are many recent development of recommendation techniques understanding its vital role. So, this application too integrates one of the Machine learning framework called Apache Mahout for recommendation system\cite{LU201512}.
\end{enumerate}  

{\bf \large  Evaluation}  \newline
The project is evaluated to identify and resolve the bugs and make desired flawless results.
Two types of testing was performed - one at the user level which is called\textbf{ Usability testing} or \textbf{Observation Testing}. And the other at the back-end level to ensure the logical functionalities using the industry based testing approach called the TDD -\textbf{ Test Driven Development}.
Both the test approaches are useful in figuring out the errors and help them locate and resolve more effectively.

{\bf  \large  Report Structure}  \newline 
The report follows the standard template which covers the \textbf{introduction} and \textbf{motivation} which overall explains the project idea and the reasons behind doing this project. Then the \textbf{literature survey} which provides evidences of existing news applications and the \textbf{proposed system} which shows some adds-on to the existing ones came across the literature survey. \newline
The in-depth report structure starts with the \textbf{requirements gathering} section which explains the stakeholders different levels of requirements and \textbf{technical specifications} gives enough idea on the development technology.\newline
Going forward the project \textbf{design phase} gives the ideas and concepts for developing the application with notes on choice of various technologies or concepts used and the \textbf{implementation phase} with brief inputs on the code structure and technical solutions applied. 
To make it complete, the project explains the testing strategies applied and the results and fixes happened in the \textbf{evaluation} section. Additionally the report provides the \textbf{challenges faced} in development process and observed challenges for the future enhancements. Due to time constraints now, the report proposes possible enhancements that could be made in the future and make the application transform to a business product in the market. Finally the report summarises the content by reflecting the \textbf{overall achievement} of the project.

 

\section{Motivation}

\begin{itemize}
\item 
The main motivation in developing the personalized buzzfeed application is to gather various news data across the countries for its end users. The application is developed keeping in mind of two major category of stakeholders of this system.
The \textbf{one} being the general casual news readers who are interested in various category news data. The \textbf{second} user-group being journalists, news companies or any organization who are interested in knowing popular category-wise locality data to better study the statistics of news by country or category.

\item

The other main motivation of this application is that to get a better understanding of how web personalization is being automated and achieved instead of one-size-fits-all approach. 
\textbf{Personalization} is now the \textbf{key-role} in promoting any kind of websites and business to its end users and knowing this standard way of automation would benefit in the long-run learning and development. This system has implemented well-formed personalization effectively achieved through two dimensions of web-personalization. 

\end{itemize}  


% \section{Description of the work}

% Explaining what your project is meant to achieve, how it is meant to function, perhaps even a functional specification.


